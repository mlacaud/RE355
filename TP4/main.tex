\documentclass[a4paper,11pt]{exam}
\usepackage[utf8]{inputenc}
\usepackage{enumerate}

%\usepackage{lipsum}

\date{\today}
\title{Introduction to Cloud Networking session 4: \\
Introduction to advanced issues in cloud computing}
\author{Simon Da Silva \and Mathias Lacaud}

%\usepackage{fancyhdr}
\usepackage{listings}
\usepackage{graphicx}

%\pagestyle{fancy}
%\fancyhf{}
\rhead{ENSEIRB-MATMECA}
\lhead{RE355 Introduction to Cloud Networking}


\lstset{language=sh,basicstyle=\ttfamily,columns=fullflexible}
\begin{document}
	
	

\maketitle

\section{Introduction}
This lab session deals with the utilisation of Docker Swarm and additional 
tools in order to discover the issues with file sharing and elasticity in 
distributed systems. As it is the last lab session of the module, the student 
are supposed to know the basics and to be able to look for informations in 
the official documentations. The student are still expected to work in 
groups of 4 with 3 to 4 computers on the same local network. 

First of all, reinitialize the computer:

\begin{lstlisting}[frame=single,language={sh}]  % Start your code-block

$ razrezo

$ date --set "$(curl https://time.akamai.com/?iso)"

$ docker service restart 

\end{lstlisting}

Be sure that chromium is installed in at least one of your computers: 

\begin{lstlisting}[frame=single,language={sh}]  % Start your code-block

$ apt install chromium
	
\end{lstlisting}

You can start chromium as root in a terminal by using: 

\begin{lstlisting}[frame=single,language={sh}]  % Start your code-block

$ chromium --no-sandbox
		
\end{lstlisting}

Then, create your swarm with: 
\begin{lstlisting}[frame=single,language={sh}]  % Start your code-block

$ docker swarm init
\end{lstlisting}

Download the correction of the 2nd lab session and extract it. Go into 
the extracted folder and start the application

\begin{lstlisting}[frame=single,language={sh}]  % Start your code-block

$ docker stack deploy --compose-file docker-compose.yml {name_of_the_stack}
\end{lstlisting}

After a few seconds, you should be able to see your running application at the address \textbf{http://127.0.0.1:3000/} in chromium.




\section{The file systems in the application}

Two modules of the application are storing informations on their file system. 
Firstly, the \textbf{Server} is storing the video to encode and the encoded video
 to be served in its folder \textbf{/go/app/fileServers}. Secondly, the \textbf{Storage}
 (i.e. the database) is storing the data in its folder \textbf{/var/lib/mysql}. 

 \begin{figure}[!ht]
	\centering
	\includegraphics[width=0.8\textwidth]{fig/architecture.png}
	\label{fig:architecture}
\end{figure}

\section{The problem with volumes...}

This section introduces the problem of volumes.




\subsection{Safe replication}
\begin{questions}
	\question Using \textbf{docker service}, create 3 replicas of the frontend.
	\begin{enumerate}[(a)]
		\item Refresh the web page. What do you see? Does the application still work?
		\item Why is the replication of the Frontend working?
	\end{enumerate}
\end{questions}

\subsection{Critical replication}
\begin{questions}
	\question Using \textbf{docker service}, create 3 replicas of the server.
	\begin{enumerate}[(a)]
		\item Refresh the web page several times. Is the application still working every time?
		\item Knowing that the video data are stored on the file system of the server, what is happening?
		\item Docker Swarm does not include any tool to create distributed volumes. Do you know any tool to create distributed file systems?
	\end{enumerate}
\end{questions}

\subsection{Persistence in the database}
\begin{questions}
	\question Using \textbf{docker service}, create 3 replicas of the database.
	\begin{enumerate}[(a)]
		\item Refresh the web page several times. Is the application still working every time?
		\item Why?
	\end{enumerate}
\end{questions}

\section{Network file system}



\section{Bonus Section: synchronizing volumes with another solution}

\end{document}
