\documentclass[a4paper,11pt]{exam}
\usepackage[utf8]{inputenc}
\usepackage{enumerate}

%\usepackage{lipsum}

\date{\today}
\title{Introduction to Cloud Networking session 3: \\
\textit{Dockerizing} Applications on multi-host environments}
\author{Simon Da Silva \and Mathias Lacaud}

%\usepackage{fancyhdr}
\usepackage{listings}
\usepackage{graphicx}

%\pagestyle{fancy}
%\fancyhf{}
\rhead{ENSEIRB-MATMECA}
\lhead{RE355 Introduction to Cloud Networking}


\lstset{language=sh,basicstyle=\ttfamily,columns=fullflexible}
\begin{document}
	
		
	
\maketitle

\section{Introduction}
This lab session deals with the utilisation of Docker Swarm in order to manage containers on multiple computers.

\section{Docker Swarm Nodes}
\subsection{Create nodes}


\subsection{Labels}

\section{Service}


\begin{lstlisting}[frame=single,language={sh}]  % Start your code-block



\end{lstlisting}

\section{Overlay Network} 

https://docs.docker.com/network/network-tutorial-overlay/

\section{Stack}

\section{Accessing your application: Load balancing}

\section{Rolling Updates}


\section{Appendix}

The benefits of the cloud and scaling, by CommitStrip\footnote{https://www.commitstrip.com}
\begin{center}
	\includegraphics[width=10cm]{fig/commitstripcloudscale.jpg}	
\end{center}

\end{document}
