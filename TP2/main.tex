\documentclass[a4paper,11pt]{exam}
\usepackage[utf8]{inputenc}
\usepackage{enumerate}

%\usepackage{lipsum}

\date{\today}
\title{Introduction to Cloud Networking session 2: \\
\textit{Dockerizing} Distributed Applications}
\author{Simon Da Silva \and Mathias Lacaud}

%\usepackage{fancyhdr}
\usepackage{listings}
\usepackage{graphicx}

%\pagestyle{fancy}
%\fancyhf{}
\rhead{ENSEIRB-MATMECA}
\lhead{RE355 Introduction to Cloud Networking}


\lstset{language=sh,basicstyle=\ttfamily,columns=fullflexible}
\begin{document}
	
		
	
\maketitle


\section{Introduction}
\subsection{Installing docker compose}
Install Docker-Compose with:

\begin{lstlisting}[frame=single,language={sh}]  % Start your code-block

$ sudo curl -L "https://github.com/docker/compose/releases/download/1.22.0/
docker-compose-$(uname -s)-$(uname -m)" -o /usr/local/bin/docker-compose

$ sudo chmod +x /usr/local/bin/docker-compose

$ docker-compose --version
\end{lstlisting}

or alternatively, with \textit{pip}: 

\begin{lstlisting}[frame=single,language={sh}]  % Start your code-block

$ pip install docker-compose

$ docker-compose --version
\end{lstlisting}


\subsection{Basic introduction to Dockerfiles}

A Dockerfile describes the software that is “baked” into an image. It isn’t just ingredients tho, it can tell the software what environment to use or what commands to run. Your recipe is often going to be very short.

Every Dockerfile should look like the following. You can find this basic example in the \textit{intro} folder.
\begin{itemize}

	\item The FROM
		\begin{lstlisting}[frame=single,language={sh}]
FROM python:3.5
	\end{lstlisting}
	The FROM keyword tells Docker which image your image is based on. In this example, you start from an ubuntu distribution. During the practical, you will have to select an appropriate base image for your language (ex: node, go, python, etc.).

	
	\item You can RUN some code to pre-install some libraries, create folders, etc. The RUN commands are executed during the build, and their effect are already here when the container is starting.

\begin{lstlisting}[frame=single,language={sh}]
RUN apt-get update -y
RUN apt-get install curl
RUN mkdir lookAtMyDirectory
\end{lstlisting}

\item The COPY, to copy the code into the container

\begin{lstlisting}[frame=single,language={sh}]
COPY . /mycode
\end{lstlisting}

	\item You can also specify what program to execute by default when the user runs a container based on this image. The CMD is not run during the build, but is the command executed when the container is starting

\begin{lstlisting}[frame=single,language={sh}]
CMD python /mycode/run.py
\end{lstlisting}	
	
	\item At the end, a Dockerfile should look like this one (but it should do something better than just printing "Coucou" ;) 
	
\begin{lstlisting}[frame=single,language={sh}]
FROM python:3.5

RUN apt-get update -y
RUN apt-get install curl
RUN mkdir lookAtMyDirectory

COPY . /mycode

CMD python /mycode/run.py
\end{lstlisting}	

\end{itemize}
At this point, you have all your software ingredients and behaviors described in a Dockerfile. You are ready to build a new image and to run a container using this image.

\begin{lstlisting}[frame=single,language={sh}]
$ docker build -t introRE355 .

$ docker run -it --rm introRE355
\end{lstlisting}	

\subsection{Basic Introduction to Docker Compose}

A basic \textit{docker-compose.yml} file is available in the folder \textit{intro}. You can see in this docker-compose.yml file that you can create "services". The services are containers. They have an "image", and they can have some other arguments like "volumes", "environment", "network", etc.

\subsection{Some Docker compose commands}

Start a compose and see the ouputs in the current terminal:

\begin{lstlisting}[frame=single,language={sh}]  % Start your code-block
	
$ docker-compose up
\end{lstlisting}

Start a compose in the background:

\begin{lstlisting}[frame=single,language={sh}]  % Start your code-block
	
$ docker-compose up -d
\end{lstlisting}

Stop all of the containers:

\begin{lstlisting}[frame=single,language={sh}]  % Start your code-block

$ docker-compose stop
\end{lstlisting}

Remove all of the containers:

\begin{lstlisting}[frame=single,language={sh}]  % Start your code-block

$ docker-compose rm
\end{lstlisting}


\subsection{Online documentation}
Additional information on docker compose can be found in the official online documentation~\footnote{https://docs.docker.com/compose/overview/}. Particularly,  in the compose file reference~\footnote{https://docs.docker.com/compose/compose-file/}.

\section{Dockerizing an application}

The idea of this TP is to deploy an application using Docker containers.
Currently, the application cannot be executed using docker. Your job is to ease the deployment of its components by using docker containers.
A high-level view of the applications is depicted in Figure~\ref{fig:architecture}.

\begin{figure}[!ht]
	\centering
	\includegraphics[width=0.8\textwidth]{fig/architecture.png}
	\label{fig:architecture}
\end{figure}

This application has five components that can be deployed in dockers containers.
Your task is to create a docker image for each component and a docker compose file to easily deploy the application.

Along the TP you will use the source code of the application. The code is in the \textbf{application} folder.

\subsection{Broker}

\begin{questions}
	\question The \textit{\textbf{Broker}} is simply a RabbitMQ server. Find and image in \textit{Docker Hub}~\footnote{https://hub.docker.com/explore/} and pull it.
	\begin{enumerate}[(a)]
		\item What is RabbitMQ?
		\item What is the standard port used in RabbitMQ?
		\item Create a \textit{compose} file and define a service ``broker'' based on the rabbitmq image.
	\end{enumerate}
\end{questions}
\subsection{Frontend}
\begin{questions}
	\question Define a Dockerfile for the \textit{\textbf{frontend}}.
	\begin{enumerate}[(a)]
		\item What base image can we use?
		\item What tools do you use to install the dependencies?
		\item Add the service to the \textit{compose} file.
	\end{enumerate}
Start your docker container. You should see in the logs: 
\begin{lstlisting}[frame=single,language={sh}]  % Start your code-block

ERROR
CORRECT USAGE: node index.js <server_ip>

Without the ip of the server, the frontend is useless by itself	
\end{lstlisting}
\begin{enumerate}[(d)]
		\item What is this error in the application? Fix it in the Dockerfile.
\end{enumerate}
	
	\textbf{Hints}:
	\begin{itemize}
		\item The application has been developped with NodeJS version 8.
		\item Remember that a \textit{Dockerfile} has: base image, copy source code, dependencies, entry point.
		\item The frontend is accessed through port 3000. You can access it using a browser.
		\item The dependencies must be installed by using \textit{npm install} 
		\item The server is executed using \textit{node index.js}.
		\item In this section and to encourage you to read the \textit{Hints}, the teachers won't answer your questions unless they begin with "Docker c'est génial, mais..."
	\end{itemize}
	\end{questions}
\subsection{Server}
	\begin{questions}

	\question Now it is time to dockerize the \textit{\textbf{server}}. Jump into the directory \textit{Server}. This component is far more complex than the \textit{\textbf{frontend}}. It is developped in Go (also called \textit{golang}) version 1.9. The component receives requests on port 8085.
	
	\begin{enumerate}[(a)] % (a)
		\item What base image can we use?
		\item Add the service to the \textit{compose} file.
		\item What are the modifications required in the previously defined services?
		\item How do you force the server to depend on the broker?
	\end{enumerate}
	Start your docker container. You should see in the logs something like: 
	\begin{lstlisting}[frame=single,language={sh}]  % Start your code-block
	
2018/10/30 16:21:09 dial tcp: lookup db on 192.168.1.142:53: no such host
exit status 1
		
	\end{lstlisting}
	\begin{enumerate}[(e)]
		\item Why the server fails during start-up?
	\end{enumerate}
	
	\textbf{Hints}:
	\begin{itemize}
		\item Find the ``right'' base image by searching in Docker Hub.
		\item Read the README.md of the Server, because the integration is very easy when you have some README :)
		\item \textit{"I do not understand why my server is not working. It should just connect to the database and... Wait ! Where is the database ?"} - a random student last year
	\end{itemize}
	\end{questions}


\subsection{Database}
	\begin{questions}
		\question The \textbf{server} needs a Mysql database version 5.6 to work. Let's dockerize the \textbf{storage}.
	
	\begin{enumerate}[(a)] % (a)
		\item What base image can we use?
		\item What is the default port of a mysql server?
		\item Add the service to the \textit{compose} file.
		\item What are the modifications required in the previously defined services?
	\end{enumerate}
		\textbf{Hints}:
	\begin{itemize}
		\item Find the ``right'' base image by searching in Docker Hub.

		\item The Server tries to connect to a database \textbf{re355}, with a username \textbf{re355} and a password \textbf{re355}.

		\item Remember that the environment variable MYSQL\_ROOT\_PASSWORD is mandatory.

		\item On the server side, use the \textbf{-d \textit{database\_addr}} argument in your CMD.

	\end{itemize}
	\end{questions}
	
\subsection{Waiting on startup}	
	\begin{questions}

	\question Sometimes, an application running within a container must wait until another container is ``ready'' (e.g., the web application of the introduction requires access to the database). However, when \textit{compose} deploys many services they are executed at the same time, without waiting. As a consequence, a container can fail because its dependencies are not ready. In the official documentation a solution is proposed~\footnote{https://docs.docker.com/compose/startup-order/}.
	
		\begin{enumerate}[(a)] % (a)
			\item What is wrong with the proposed solution?
			\item How should you modify the \textit{\textbf{Server}}'s \textit{dockerfile} to fix this issue? Hints: you have a wait-for-it.sh file in the Server sources.
		\end{enumerate}
	\end{questions}
\subsection{Worker and full application}
	\begin{questions}
	
	\question Use file \textit{Worker/Dockerfile} to create the worker image. Afterwards, complete the \textit{composition} and deploy it. You can use the archive \textit{trailer.zip} in the frontend to answer the following questions:
	
	\begin{enumerate}[(a)] % (a)
		\item After uploading a video file, the results files are stored in which server? 
		\item Show your working application to your teacher. Who knows, he may give you a cookie... 
	\end{enumerate}
	
	\textbf{Hint:}
	\begin{itemize}
	\item The Worker need an environment variable \textbf{BROKER\_ADDR}. You can add it with \textit{-e BROKER\_ADDR=localhost:5672}
	\item You need to reload the webpage to update the list of available videos. 
	\end{itemize}
	\end{questions}


\subsection{Improving image size}	
	\begin{questions}

	\question By using \textit{docker images}, you can see the size of the images. The images are big because they are embedding all of the dependencies. However, as the Server image is written in Go, a langage that can be compiled, there is something we can do...
	
	\begin{enumerate}[(a)] % (a)
		\item In the Server folder, run an interactive container with a golang image. Bind a folder of the container with the current folder and indicate the command line you have used. (Hints: Take a look at the \textbf{-v} option of docker).
		\item Build the go project using the following line, and take a look at the folder. Can you see a new file?

		\begin{lstlisting}[frame=single,language={sh}]  % Start your code-block

			go build -tags netgo main.go

		\end{lstlisting}
		\item Change you Dockerfile to use \textit{alpine} as a base image. Copy the new file and run it. Build the new image. Take alook at its size. Conclude. 
	\end{enumerate}
\end{questions}

\section{Appendix}

At the end of this lab session, you will be rewarded by a beautiful red panda !
\begin{center}
	\includegraphics[width=12cm]{fig/redpanda.png}	
\end{center}

\end{document}
